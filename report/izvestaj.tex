\documentclass[conference]{IEEEJERM}
% Some Computer Society conferences also require the compsoc mode option,
% but others use the standard conference format.
%
% If IEEEtran.cls has not been installed into the LaTeX system files,
% manually specify the path to it like:
% \documentclass[conference]{../sty/IEEEtran}

%\usepackage[utf8]{inputenc}
\usepackage[T1, T2A]{fontenc}
\usepackage[serbianc]{babel}

\usepackage{listings}
\usepackage{xcolor}
\usepackage{xparse}

\usepackage{hyperref}

\usepackage{graphicx}
\usepackage{float}
\graphicspath{{../slike/}}

\renewcommand{\familydefault}{\sfdefault}

\NewDocumentCommand{\codeword}{v}{%
\texttt{\textcolor{black}{#1}}%
}

\lstset{language=python,keywordstyle={\bfseries \color{black}}}

% *** MATH PACKAGES ***
%
\usepackage{amsmath}

% *** ALIGNMENT PACKAGES ***
%
\usepackage{array}

\usepackage{siunitx} % for micro meters



% *** SUBFIGURE PACKAGES ***
\ifCLASSOPTIONcompsoc
  \usepackage[caption=false,font=normalsize,labelfont=sf,textfont=sf]{subfig}
\else
  \usepackage[caption=false,font=footnotesize]{subfig}
\fi


% *** PDF, URL AND HYPERLINK PACKAGES ***
%
\usepackage{url}
% url.sty was written by Donald Arseneau. It provides better support for
% handling and breaking URLs. url.sty is already installed on most LaTeX
% systems. The latest version and documentation can be obtained at:
% http://www.ctan.org/pkg/url
% Basically, \url{my_url_here}.


% correct bad hyphenation here
\hyphenation{op-tical net-works semi-conduc-tor}


\begin{document}
%
% paper title
% Titles are generally capitalized except for words such as a, an, and, as,
% at, but, by, for, in, nor, of, on, or, the, to and up, which are usually
% not capitalized unless they are the first or last word of the title.
% Linebreaks \\ can be used within to get better formatting as desired.
% Do not put math or special symbols in the title.
\title{Систем за директну дигиталну синтезу учестаности}


% author names and affiliations
% use a multiple column layout for up to three different
% affiliations

\author{
\IEEEauthorblockN{Александар Арсовић, студент}
\IEEEauthorblockA{Електротехнички факултет\\
Универзитет у Београду}
\and
\IEEEauthorblockN{Александар Вуковић, студент}
\IEEEauthorblockA{Електротехнички факултет\\
Универзитет у Београду}}


% make the title area
\maketitle

% As a general rule, do not put math, special symbols or citations
% in the abstract


\IEEEpeerreviewmaketitle



\section{Увод}

Циљ овог пројекта је софтверска имплементација система за директну дигиталну синтезу учестаности.
Пројекат је урађен у програмском језику \texttt{Python}.
Захтеви за овај систем су следећи:

\begin{enumerate}
	\item резолуција система: $\Delta f = \SI{100}{\micro \hertz}$,
	\item генерисана синусоида  у опсегу од $f = \left[ \SI{100}{\micro \hertz}, \SI{40}{\mega \hertz} \right]$,
	\item спектралне реплике потиснуте бар \SI{60}{\decibel},
	\item варијација амплитуде у опсегу $\pm \SI{0.05}{\decibel}$,
	\item генерисање сигнала у првој и трећој Никвистовој зони.
\end{enumerate}


\section{Ширина контролне речи фазног акумулатора}

Ширина контролне речи $W$ како би резолуција била $ \Delta f = \SI{100}{\micro\hertz}$ за сигнал такта $f_{clk} = \SI{100}{\mega\hertz}$ износи.

\begin{equation}
f_0 = \dfrac{M f_{clk}}{2^W}
\end{equation}

\begin{equation}
W = \log_2 \left(\dfrac{M f_{clk}}{ f_0}\right)
\end{equation}

\begin{equation}
W = \log_2 \left(\dfrac{\SI{100}{\mega \hertz}}{\SI{100}{\micro \hertz}}\right) = 39.863 
\end{equation}

\begin{equation}
W \approx 40
\end{equation}

од тога су два бита за квадрант, а остали за вредности унутар једног квадранта.

Фазни акумулатор ради тако што генерише одбирке фазе у опесгу од $\left[-\pi, \pi \right]$.
Излаз фазног акумулатора приказан је на слици \ref{fig:fai}.

\begin{figure}[H]
	\centering
	\includegraphics[width=0.9\linewidth]{fazni_akumulator_izlaz.pdf}
	\caption{Излаз фазног акумулатора}
	\label{fig:fai}
\end{figure}

До додатне грешке у квантизацији долази и после смањења броја бита са 40 на 14, односно после генератора одбирака.
Да би се смањио утицај квантизационог шума, у фазном акумулатору остављена је опција за укључивање несупстрактивног дитеринга.


\section{Архитектура генератора одбирака}

За генерисање одбирака  $cos(x)$ коришћен је CORDIC алгоритам чиме се постиже добар компромис између прецизности, количине потребне меморије и времена извршавања.	

На сликама \ref{fig:cordic_sample} и \ref{fig:cordic_fft} приказани су временски и фреквенцијски облик сигнал на излазу генератора одбирака.

\begin{figure}[t]
	\centering
	\includegraphics[width=0.9\linewidth]{cordic_sample.pdf}
	\caption{Временски облик сигнала на излазу генератора одбирака}
	\label{fig:cordic_sample}
\end{figure}


\begin{figure}[t]
	\centering
	\includegraphics[width=0.9\linewidth]{cordic_fft.pdf}
	\caption{Спектар сигнала на излазу генератора одбирака}
	\label{fig:cordic_fft}
\end{figure} 


% \subsection{Сложеност имплементације}


\section{FIR филтар}
FIR филтар се користи за компензацију кола задршке нултог реда чија је фреквенцијска карактеристика $sin(x)/x$.

За одређивање коефицијената филтра коришћена је \codeword{scipy.signal.firls()} која за задати ред филтра, 
фреквенцијски опсег и фреквенцијску карактеристику израчунава потребне коефицијенте. 
Филтар 6. реда задовољава услов из пројекта да варијација амплитуде излазног сигнал буде $\pm \SI{0.05}{\decibel}$.

Највеће одступање амплитуде је на $\SI{40}{\mega\hertz}$ и износи $\SI{0.035}{\decibel}$.

На сликама \ref{fig:fir} и \ref{fig:ampl_var} приказане су карактеристика FIR филтра преклопљена са $x/sin(x)$ и варијација амплитуде у зависности од фреквенције. 

\begin{figure}
	\centering
	\includegraphics[width=0.9\linewidth]{sinx_korekcija.pdf}
	\caption{Карактеристика FIR филтра}
	\label{fig:fir}
\end{figure}

\begin{figure}
	\centering
	\includegraphics[width=0.9\linewidth]{varijacija_amplitude_fir.pdf}
	\caption{Варијација амплитуде}
	\label{fig:ampl_var}
\end{figure}


\section{Аналогни филтар}

За потискивање спектралних копија у вишим Никвистовим зонама користи се нископропусни филтар.
Користи се Чебишевљев филтар друге врсте, због његове равне карактеристике у пропусном опсегу.
Захтев да спектралне реплике буду потиснуте бар 60dB је постигнут филтром 11. реда. 
Гранична учестаност пропусног опсега је $\SI{40}{\mega\hertz}$, док је гранична учестаност непропусног опсега $\SI{60}{\mega\hertz}$, 
због тога што знамо да ту неће постојати спектраллне компоненте сигнала. Фреквенцијска карактеристика приказана је на слици \ref{fig:lp}.

\begin{figure}[t]
  \centering
  \includegraphics[width=0.9\linewidth]{lp_fft.pdf}
  \caption{Фреквенцијска карактеристика нископропусног филтра}
  \label{fig:lp}
\end{figure}

\section{Потискивање спурова услед квантизације фазе и амплитуде}

У овом систему ДА конвертор направљен је тако да квантизује сигнал да би се прецизније симулирао квантизациони шум.
Квантизација је рађена на два начина: коришћењем квантизатора из материјала са вежби и коришћењем библиотеке гита \href{https://github.com/rwpenney/spfpm/}{rwpenney/spfpm/}.

Због детерминистичке природе квантизационог шума пре ДА конвертора на сигнал се додаје несупстрактивни дитер стохастичке природе и ствара апериодичне сигнале слабљењем (разбијањем) периодичних компоненти квантизационог шума.

Модификација система је додатни сабирач и генератор псеудо случајних бројева. На слици \ref{fig:da_fft} је приказан спректар сигнала на излазу ситстема.
На овој слици види се и утицај нископропусног филтра, који потискује све нежељене фреквенције према задатим спецификакцијама. 
На слици \ref{fig:da_ifft} приказан је временски облик излазног сигнала.

\begin{figure}[t]
	\centering
	\includegraphics[width=0.9\linewidth]{da_fft.pdf}
	\caption{Спектар излазног сигнала}
	\label{fig:da_fft}
\end{figure}

\begin{figure}[t]
	\centering
	\includegraphics[width=0.9\linewidth]{da_ifft.pdf}
	\caption{Временски облик излазног сигнала}
	\label{fig:da_ifft}
\end{figure}


\section{Максимални џитер такта}

Када се узме у обзир само шум који потиче од џитера такта тада је однос сигнал-шум:

\begin{equation}
SNR=20\log10\frac{1}{2\pi f t_j}.
\end{equation}


Ако се посматра само допринос шума услед квантизације тада је $SNR = 6.02N + 1.76$, онда је максимални џитер такта који однос сигнал-шум не деградира више од квантизационог шума:

\begin{equation}
t_j = \frac{1}{2 \pi f_{max}} 10^{-\frac{6.02N +1.76}{20}}
\end{equation}

\begin{equation}
t_j \approx  \SI{0.2}{\pico\s}
\end{equation}

\noindent где је $N$ број бита конвертора,  $f_{max}$ максимална улазна учестаност у конвертор.

\section{Одабирање у трећој Никвистовој зони}

Одабирањем сигнала у трећој Никвистовој зони можемо да остваримо веће излазне учестаности у односу на учестаност одабирања.
 У спектру излазног сигнала после ДА конвертора јављају се спектралне реплике сигнала и филтрирањем сигнала 
 у опсегу од $f_s$ до $\frac{3}{2} f_s $ генеришемо сигнал изнад учестаности одабирања.

Коло задршке нултог реда има лошу карактеристику у трећој Никвистовој зони, па га је потребно променити. 
Изабрано је биполарно коло задршке нултог реда са повратком на нулу, које у трећој Никвистовој зони има равну фреквенцијску карактеристику.

Нископропусни филтар треба да се замени филтром пропусника опсега учестаности. 
Прелазне зоне за тај филтар су од $\SI{90}{\mega\hertz}$ до $\SI{100}{\mega\hertz}$ и од $\SI{140}{\mega\hertz}$ до $\SI{160}{\mega\hertz}$.
Прва прелазна зона је ужа да би сигнал из друге Никвистове зоне што мање утицао на користан сигнал.
Идеално би било када би она била бесконачно уска, да би сигнали, који су на ниским учестаностима првој зони што мање утицали на излазни сигнал,
јер се они у другој зони налазе на вишим учестаностима. Како би за то био потребан бесконачан ред филтра, остварен је компромис између реда филтра
и ширине прелазне зоне.


Израчунати џитер за систем са одабирањем у трећој Никвистовој зони:

\begin{equation}
t_{j} = \SI{56}{\femto\second}
\end{equation}

На сликама \ref{fig:bp_fft}, \ref{fig:da_brz_fft} и \ref{fig:da_brz_ifft} приказане су фреквенцијска карактеристика филтра пропусника опсега,
спектар излазног сигнала у трећој Никвистовој зони и временски облик тог сигнала.

\begin{figure}[t]
	\centering
	\includegraphics[width=0.9\linewidth]{bp_fft.pdf}
	\caption{Фреквенцијска карактеристика филтра пропусника опсега}
	\label{fig:bp_fft}
\end{figure}

\begin{figure}[t]
	\centering
	\includegraphics[width=0.9\linewidth]{da_brz_fft.pdf}
	\caption{Спектар излазног сигнала у трећој Никвистовој зони}
	\label{fig:da_brz_fft}
\end{figure}

\begin{figure}[t]
	\centering
	\includegraphics[width=0.9\linewidth]{da_brz_ifft.pdf}
	\caption{Временски облик излазног сигнала у трећој Никвистовој зони}
	\label{fig:da_brz_ifft}
\end{figure}



\section{Закључак}


У имплементацији овог система коришћени бројеви са ограниченом прецизношћу да би се боље симулирао квантизациони шум.
То је урађено тако што су се после сваког израчунавања у коду бројеви квантизовали на одређен број бита.
За ограничавање прецизности коришћен је квантизатор из материјала са вежби и библиотека са гита \href{https://github.com/rwpenney/spfpm/}{rwpenney/spfpm/}.
Квантизациони шум изражен је у спектру за мању резолуцију ДА конвертора, док за конвертор са 14 бита има довољну прецизност, те шум није изражен,
па закључујемо да у овом систему нема потребе за дитерингом. 




\end{document}


