\documentclass[conference]{IEEEJERM}
% Some Computer Society conferences also require the compsoc mode option,
% but others use the standard conference format.
%
% If IEEEtran.cls has not been installed into the LaTeX system files,
% manually specify the path to it like:
% \documentclass[conference]{../sty/IEEEtran}

\usepackage[T1, T2A]{fontenc}
\usepackage[serbianc]{babel}

\usepackage{listings}
\usepackage{xcolor}
\usepackage{xparse}

\NewDocumentCommand{\codeword}{v}{%
\texttt{\textcolor{blue}{#1}}%
}

\lstset{language=python,keywordstyle={\bfseries \color{blue}}}

% *** MATH PACKAGES ***
%
\usepackage{amsmath}

% *** ALIGNMENT PACKAGES ***
%
\usepackage{array}

\usepackage{siunitx} % for micro meters



% *** SUBFIGURE PACKAGES ***
\ifCLASSOPTIONcompsoc
  \usepackage[caption=false,font=normalsize,labelfont=sf,textfont=sf]{subfig}
\else
  \usepackage[caption=false,font=footnotesize]{subfig}
\fi


% *** PDF, URL AND HYPERLINK PACKAGES ***
%
\usepackage{url}
% url.sty was written by Donald Arseneau. It provides better support for
% handling and breaking URLs. url.sty is already installed on most LaTeX
% systems. The latest version and documentation can be obtained at:
% http://www.ctan.org/pkg/url
% Basically, \url{my_url_here}.


% correct bad hyphenation here
\hyphenation{op-tical net-works semi-conduc-tor}


\begin{document}
%
% paper title
% Titles are generally capitalized except for words such as a, an, and, as,
% at, but, by, for, in, nor, of, on, or, the, to and up, which are usually
% not capitalized unless they are the first or last word of the title.
% Linebreaks \\ can be used within to get better formatting as desired.
% Do not put math or special symbols in the title.
\title{Систем за директну дигиталну синтезу учестаности}


% author names and affiliations
% use a multiple column layout for up to three different
% affiliations
\author{\IEEEauthorblockN{Александар Арсовић}
\IEEEauthorblockA{Електротехнички факултет\\
Универитет у Београду}}

\hyphenation{}

% make the title area
\maketitle

% As a general rule, do not put math, special symbols or citations
% in the abstract
\begin{abstract}
The abstract goes here.
\end{abstract}

\IEEEpeerreviewmaketitle



\section{Увод}

% \hfill mds ??
 
% \hfill August 26, 2015


\section{Ширина контролне речи $f_0$ фазног акумулатора}

Ширина контролне речи $W$ како би резолуција била $ \Delta f = \SI{100}{\micro\hertz}$ за сигнал такта $f_{clk} = \SI{100}{\mega\hertz}$ износи.

\begin{equation}
f_0 = \dfrac{M * f_{clk}}{2^W}
\end{equation}

\begin{equation}
W = log2\left(\dfrac{ M*f_{clk} }{ f_0}\right)
\end{equation}

\begin{equation}
W = log2\left(\dfrac{ 1 * 100 M Hz }{ 100 \mu Hz}\right) = 39.863 
\end{equation}

\begin{equation}
W \approx 40
\end{equation}

од тога су два бита за квадрант, а остали за нпр. LUT (енг. \textit{Look Up Table}).

до грешке квантизације због тога што број потребних битова контролне речи већи од броја битова ДА конвертора.

несупстрактивни дитеринг

\section{Архитектура генератора одбирака}

За генерисање одбирака  $cos(x)$ коришћен је CORDIC алгоритам. 


\subsection{Могуће оптимизације}

\subsection{Сложеност имплементације}


\section{FIR филтар}
FIR филтар се користи за компензацију кола задршке нултог реда чија је фреквенцијска карактеристика .

За одређивање коефицијената филтра коришћена је \codeword{scipy.signal.firls()} која за задати ред филтра, фреквенцијски опсег и фреквенцијску карактеристику израчунава потребне коефицијенте. Филтар 6. реда задовољава услов из пројекта да варијација амплитуде излазног сигнал буде $\pm \SI{0.05}{\decibel}$.

Највеће одступање амплитуде је на $\SI{40}{\mega\hertz}$ и износи :::

На слици је приказана варијација амплитуде. На слици је приказана карактеристика FIR филтра преклопљена са $x/sin(x)$.
 
\section{Аналогни филтар}

За потискивање спектралних копија у вишим Никвистовим зонама користи се нископропусни филтар. Користи се Чебишевљев филтар друге врсте, због његове равне карактеристике у пропусном опсегу. Захтев је био да спектралне реплике буду потиснуте бар 60dB, што је постигнуто филтром 11. реда. Гранична учестаност пропусног опсега је $\SI{40}{\mega\hertz}$, док је гранична учестаност непропусног опсега $\SI{60}{\mega\hertz}$, због тога што знамо да ту неће постојати спектраллне компоненте сигнала.

Слика фрекв карактеристике аналогног лоупаса


\section{Потискивање спурова услед квантизације фазе и амплитуде}

Због детерминистичке природе квантизационог шума пре ДА конвертора на сигнал се додаје несупстрактивни дитер стохастичке природе и ствара апериодичне сигнале слабљењем (разбијањем) периодичних компоненти квантизационог шума.

Модификација система је додатни сабирач и генератор псеудо случајних бројева.

слика спектра са потискивањем

слика спектра без потискивањем

\section{Максимални џитер такта}

Када се узме у обзир само шум који потиче од џитера такта тада је однос сигнал-шум:

\begin{equation}
SNR=20\log10\frac{1}{2\pi f t_j}.
\end{equation}


Ако се посматра само допринос шума услед квантизације тада је $SNR = 6.02N + 1.76$, онда је максимални џитер тактка који однос сигнал-шум не деградира више од квантизационог шума:
\begin{equation}
t_j = \frac{1}{2 \pi f_{max}} 10^{-\frac{6.02N +1.76}{20}}
\end{equation}


\begin{equation}
t_j \approx  \SI{0.2}{\pico\s}
\end{equation}

\noindent где је $N$ број бита конвертора,  $f_{max}$ максимална улазна учестаност у конвертор.

\section{Одабирање у трећој Никвистовој зони}

Одабирањем сигнала у трећој Никвистовој зони можемо да остваримо веће излазне учестаности у односу на учестаност одабирања. У спектру излазног сигнала после ДА конвертора јављају се спектралне реплике сигнала и филтрирањем сигнала у опсегу од $f_s$ до $\frac{3}{2} f_s $ генеришемо сигнал изнад учестаности одабирања.

Коло задршке нултог реда има лошу карактеристику у трећој Никвистовој зони па га је потребно променити. Изабрано је биполарно коло задршке нултог реда са повратком на нулу, које у трећој Никвистовој зони има равну фреквенцијску карактеристику.

Нископропусни филтар треба да се замени филтром пропусника опсега учестаности. Прелазне зоне за тај филтар су од $\SI{90}{\mega\hertz}$ до $\SI{100}{\mega\hertz}$ и од $\SI{140}{\mega\hertz}$ до $\SI{160}{\mega\hertz}$, јер у тим опсезима сигурно нема корисног сигнала. Прва прелазна зона је ужа да би сигнал из друге Никвистове зоне што мање утицао на користан сигнал. 

генерисање нижих учестаности проблем?

Израчунати џитер за систем са одабирањем у трећој Никвистовој зони:

\begin{equation}
t_{j} = \SI{56}{\femto\second}
\end{equation}





\section{Закључак}
The conclusion goes here.




% conference papers do not normally have an appendix
% use section* for acknowledgment
\section*{Acknowledgment}


The authors would like to thank...


\begin{thebibliography}{1}

\bibitem{IEEEhowto:kopka}

\end{thebibliography}

\end{document}


