\documentclass[conference]{IEEEJERM}
% Some Computer Society conferences also require the compsoc mode option,
% but others use the standard conference format.
%
% If IEEEtran.cls has not been installed into the LaTeX system files,
% manually specify the path to it like:
% \documentclass[conference]{../sty/IEEEtran}

\usepackage[T1, T2A]{fontenc}
\usepackage[serbianc]{babel}

% *** MATH PACKAGES ***
%
\usepackage{amsmath}

% *** ALIGNMENT PACKAGES ***
%
\usepackage{array}

% \usepackage{siunitx} % for micro meters



% *** SUBFIGURE PACKAGES ***
\ifCLASSOPTIONcompsoc
  \usepackage[caption=false,font=normalsize,labelfont=sf,textfont=sf]{subfig}
\else
  \usepackage[caption=false,font=footnotesize]{subfig}
\fi


% *** PDF, URL AND HYPERLINK PACKAGES ***
%
\usepackage{url}
% url.sty was written by Donald Arseneau. It provides better support for
% handling and breaking URLs. url.sty is already installed on most LaTeX
% systems. The latest version and documentation can be obtained at:
% http://www.ctan.org/pkg/url
% Basically, \url{my_url_here}.


% correct bad hyphenation here
\hyphenation{op-tical net-works semi-conduc-tor}


\begin{document}
%
% paper title
% Titles are generally capitalized except for words such as a, an, and, as,
% at, but, by, for, in, nor, of, on, or, the, to and up, which are usually
% not capitalized unless they are the first or last word of the title.
% Linebreaks \\ can be used within to get better formatting as desired.
% Do not put math or special symbols in the title.
\title{Систем за директну дигиталну синтезу}


% author names and affiliations
% use a multiple column layout for up to three different
% affiliations
\author{\IEEEauthorblockN{Александар Арсовић}
\IEEEauthorblockA{Електротехнички факултет\\
Универитет у Београду}}



% make the title area
\maketitle

% As a general rule, do not put math, special symbols or citations
% in the abstract
\begin{abstract}
The abstract goes here.
\end{abstract}

\IEEEpeerreviewmaketitle



\section{Увод}

% \hfill mds ??
 
% \hfill August 26, 2015


\section{Ширина контролне речи $f_0$}

Ширина контролне речи $W$ да резолуција буде %$\SI{100}{\micro}$. Сигнал такта је 100  $\SI{100}{\mega\hertz}}$

\begin{equation}
f_0 = \dfrac{(M * f_{clk})}{2^W}
\end{equation}

\begin{equation}
W = log2\left(\dfrac{ M*f_{clk} }{ f_0}\right)
\end{equation}

\begin{equation}
W = log2\left(\dfrac{ 1 * 100 M Hz }{ 100 \mu Hz}\right) = 39.863 
\end{equation}

\begin{equation}
N = 40
\end{equation}

од тога су два бита за квадрант, а остали за нпр. LUT (енг. \textit{Look Up Table}).

\section{Архитектура генератора одбирака}

За генерисање одбирака  $cos(x)$ коришћен је алгоритам CORDIC ()


\section{Conclusion}
The conclusion goes here.




% conference papers do not normally have an appendix
% use section* for acknowledgment
\section*{Acknowledgment}


The authors would like to thank...


\begin{thebibliography}{1}

\bibitem{IEEEhowto:kopka}

\end{thebibliography}

\end{document}


