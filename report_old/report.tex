\documentclass[journal,twocolumn,letterpaper]{IEEEJERM}

\usepackage{times,amsmath,epsfig}
\usepackage{fancyhdr}
\usepackage{amsmath}
\usepackage{amsfonts}
\usepackage{amssymb}
\usepackage[utf8]{inputenc}
\usepackage[serbianc]{babel}
\usepackage[T1, T2A]{fontenc}

\usepackage{array}
\usepackage{graphicx}
\usepackage{url}
\usepackage{subfigure}
\usepackage{bm}
\usepackage{breqn}
\usepackage{xcolor}
\usepackage{soul}


% \hyphenation{op-tical net-works semi-conduc-tor}


\newcommand{\HRule}{\rule{\linewidth}{0.3mm}} % Defines a new command for the horizontal lines, change thickness here


\begin{document}

\title{Sistem za direktnu digitalnu sintezu}

\author{Aleksandar Vuković 2018/3034, \textit{student}}

\markboth{Univerzitet u Beogradu, Elektrotehnički fakultet, Hardversko-softverska obrada signala}
{}

\twocolumn[
\begin{@twocolumnfalse}
  
\maketitle

\begin{abstract}
Sistem za direktnu digitalnu sintezu
\end{abstract}

\end{@twocolumnfalse}]

\IEEEpeerreviewmaketitle



\section{Postavka projekta}


\section{Kontrolna reč $f_0$}
Učestanost generisanog signala se zadaje W-bitnom kontrolnom rečju $f_0$.



\section{Arhitektura generatora odbiraka \textit{cos(x)}}



\section{Fazni akumulator}


\section{FIR filtar}

Za kauzalni FIR (\textit{Finite impulse response}) filtar u diskretnom vremenu, svaka izlazna sekvenca je jednaka otežanom zbiru prethodnih vrednosti ulaza:

\begin{equation}
y[n] = b_0 x[n] + b_1 x[n-1] + ... + b_N x[n-N]
\end{equation}

FIR filtar N-tog reda ima N+1 koeficijenata.

za razliku od IIR (\textit{Infinite impulse response}) filtra gde je izlaz jednak:

\begin{equation}
y[n] =\dfrac{1}{a_0} (b_0 x[n] + ... + b_P x[n-P]) - a_1 y[n-1] - ... - a_Q y[n-Q]
\end{equation}

IIR filtar je definican pomoću P i Q

Iz ovih formula se može videti i razlika između \textbf{FIR} i \textbf{IIR} filtra:

\begin{itemize}
	\item IIR filtar ima povratnu spregu, a FIR nema, što znači da je FIR filtar inherentno stabilan
	\item FIR se može lako dizajnirati da ima linearnu fazu
\end{itemize}


\subsection{Dizajn FIR filtra}

\section{Generator odbirka}


\section{CORDIC algoritam}


\section{Ostalo}

NP problemi u računarskoj tehnici (computer science) \\

NP potpuni problemi za generisanje \textbf{MCM}-a 

\end{document}